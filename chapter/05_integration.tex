\chapter[Integrationsprozesse]{Integrationsprozesse}

Nach den behandelten Herausforderungen werden in diesem Kapitel Möglichkeiten und Vorgehensmodelle zur Integration von Data Science in Organisationen thematisiert.
Dazu werden insgesamt fünf Modelle der Forschungsliteratur dargelegt.

\section{CSPG Framework}

\Citeauthor*{Grossman.2014} veröffentlichte 2014 das CSPG Framework zur Integration von Analytik, Domänenwissen und IT in Organisationen.
\textit{CSPG} steht dabei repräsentativ als Abkürzung für die Komponenten \textit{Culture, Staffing, Processes} und \textit{Governance}. 
Der Aspekt der Kultur ist dabei durch die Organisationsleitung umzusetzen, indem Verantwortung und Authorität für Datenbestände an eine Funktionsstelle übergeben wird.
Die Aufnahme von Data Science Personal ist im CSPG Framework unausweichlich und durch den Analytikleiter und die Geschäftsleitung durchzuführen. 
Dabei ist zu entscheiden, ob die analytische Funktion zentral, dezentral oder hybrid organisiert wird.
Im dritten Aspekt des Frameworks sind die analytischen Prozesse in der Organisation aufzubauen.
Ein Teil dieser Prozesse umfasst den Austausch von Daten zwischen Abteilungen und anderen Organisationen.
Weitere Prozesse behandeln die Digitalisierung bestehender Inhalte, Produktanpassung zur Aufnahme von Daten und die Kombination von Datenbeständen mit anderen Industrien.
Die finale Komponente des Frameworks ist der Aufbau, die Verwaltung und die Weiterentwicklung der notwendigen Infrastruktur.
Zur Bewältigung dieser Aufgabe sind die folgenden vier Parameter in der Organisation einzustellen:

\begin{itemize}
    \item Langfristige Verpflichtung für Data Science und Sicherstellung des daraus entstehenden Geschäftswerts.
    \item Sichere und rechtlich unbedenkliche Umsetzung der Data Science Prozesse.
    \item Herstellen von Haftbarkeit, Transparenz und Rückverfolgbarkeit für Projektfinanzierung, Entwicklung und Ressourcen.
    \item Ressourcenbereitstellung für Daten-, Analyse- und Managementprozesse.
\end{itemize}

% insert table summary of CSPG
% \begin{table}[htb]
%     \centering
%     \begin{tabular}{>{\raggedright\arraybackslash}m{2.5cm} >{\raggedright\arraybackslash}m{5cm} >{\raggedright\arraybackslash}m{5cm}}
%     \toprule
%      & \textbf{Department/Unit Level} & \textbf{Organizational Level} \\ \midrule
%     \textbf{Analytics Culture} & Sind Big Data und Analytics als organisatorische Funktion betrachtet und gibt es eine Big Data/Analytics-Abteilung oder -Einheit, um diese Funktion zu unterstützen? & Sind Big Data und Analytics in die Unternehmensstrategie integriert? Gibt es einen Senior Leader, der sich für Big Data und Analytics einsetzt? Werden interne und externe Daten, die Wert schaffen können, genutzt? \\ \midrule
%     \textbf{Analytics Staff} & Hat die Analytics-Abteilung die richtigen Personen mit dem richtigen Grad an analytischer Spezialisierung, IT-Wissen und Geschäftswissen? & Gibt es Analytikteammitglieder in den richtigen Abteilungen innerhalb der Organisation und gibt es eine kritische Masse an analytischem Talent? Wenn nicht, muss das analytische Personal neu ausbalanciert oder die Zentralisierung/Dezentralisierung des Analysepersonals entsprechend angepasst werden. \\ \midrule
%     \textbf{Analytics Processes} & Hat die Analyseabteilung Analyseprozesse eingerichtet, um Analysemodelle zu erstellen, Analysemodelle einzusetzen und deren Geschäftsauswirkungen zu messen? & Hat die Organisation die analytischen Prozesse eingerichtet, um analytische Möglichkeiten auszuwählen, Daten für die Datenwissenschaftler bereitzustellen, analytische Modelle zu erstellen, analytische Modelle einzusetzen und den generierten Geschäftswert zu messen? Gibt es eine analytische Governance-Struktur, um die richtigen analytischen Prozesse zu unterstützen und zu koordinieren? \\ \bottomrule
%     \end{tabular}
%     \caption{Vergleich von Analytics auf Abteilungs-/Einheitsebene und organisatorischer Ebene}
% \end{table}

\section{Design Parameters}

Durch die Arbeit von \Citeauthor*{JanineAdinaHagen.2020} konnten die Gestaltungsparameter einer datengesteuerten Organisation ermittelt werden.
Diese Parameter können als Anleitung betrachtet werden, wie die eigene Organisation in verschiedenen Aspekten zu gestalten ist.
Folgende Tabelle zeigt die Gestaltungsparameter organisiert nach Komponenten, Dimensionen und konkreter Charakteristika:

% Please add the following required packages to your document preamble:
% \usepackage{multirow}
% \begin{table}[]
%     \begin{tabular}{|cc|c|cccccccc|}
%     \hline
%     \multicolumn{2}{|c|}{\textbf{\begin{tabular}[c]{@{}c@{}}Subsystem/ \\  component\end{tabular}}} & \textbf{\begin{tabular}[c]{@{}c@{}}Design \\  dimension\end{tabular}} & \multicolumn{8}{c|}{\textbf{Characteristic}} \\ \hline
%     \multicolumn{1}{|c|}{\multirow{7}{*}{Social}} & \multirow{6}{*}{\begin{tabular}[c]{@{}c@{}}Struc- \\  ture\end{tabular}} & \textit{\begin{tabular}[c]{@{}c@{}}Anchoring of \\  data experts\end{tabular}} & \multicolumn{2}{c|}{Central} & \multicolumn{4}{c|}{Hybrid} & \multicolumn{2}{c|}{Decentral} \\ \cline{3-11} 
%     \multicolumn{1}{|c|}{} &  & \textit{\begin{tabular}[c]{@{}c@{}}Reporting \\  line\end{tabular}} & \multicolumn{2}{c|}{Technology} & \multicolumn{4}{c|}{Dual} & \multicolumn{2}{c|}{Business} \\ \cline{3-11} 
%     \multicolumn{1}{|c|}{} &  & \textit{\begin{tabular}[c]{@{}c@{}}Horizontal \\  linkage\end{tabular}} & \multicolumn{1}{c|}{\begin{tabular}[c]{@{}c@{}}Simplified \\  examples\end{tabular}} & \multicolumn{1}{c|}{\begin{tabular}[c]{@{}c@{}}Meeting \\  routines\end{tabular}} & \multicolumn{1}{c|}{\begin{tabular}[c]{@{}c@{}}Joint \\  processes\end{tabular}} & \multicolumn{2}{c|}{\begin{tabular}[c]{@{}c@{}}Voluntary \\ works\end{tabular}} & \multicolumn{1}{c|}{Training} & \multicolumn{1}{c|}{Integrator} & Events \\ \cline{3-11} 
%     \multicolumn{1}{|c|}{} &  & \textit{\begin{tabular}[c]{@{}c@{}}Collaboration \\  initiative\end{tabular}} & \multicolumn{2}{c|}{Business team} & \multicolumn{2}{c|}{Data and business team} & \multicolumn{2}{c|}{\begin{tabular}[c]{@{}c@{}}Data team\\ (business need-based)\end{tabular}} & \multicolumn{2}{c|}{\begin{tabular}[c]{@{}c@{}}Data team \\  (data-based)\end{tabular}} \\ \cline{3-11} 
%     \multicolumn{1}{|c|}{} &  & \textit{\begin{tabular}[c]{@{}c@{}}Collaboration \\  mode\end{tabular}} & \multicolumn{2}{c|}{Prototyping} & \multicolumn{4}{c|}{Structured backlog} & \multicolumn{2}{c|}{Occasional use cases} \\ \cline{3-11} 
%     \multicolumn{1}{|c|}{} &  & \textit{\begin{tabular}[c]{@{}c@{}}Control \\  mechanisms\end{tabular}} & \multicolumn{1}{c|}{\begin{tabular}[c]{@{}c@{}}Business \\  impact\end{tabular}} & \multicolumn{1}{c|}{Back testing} & \multicolumn{1}{c|}{Utilization} & \multicolumn{3}{c|}{\begin{tabular}[c]{@{}c@{}}Data vs. human \\ tournament\end{tabular}} & \multicolumn{2}{c|}{\begin{tabular}[c]{@{}c@{}}Transformation \\  KPIs\end{tabular}} \\ \cline{2-11} 
%     \multicolumn{1}{|c|}{} & Actors & \textit{Roles} & \multicolumn{2}{c|}{Data-oriented} & \multicolumn{4}{c|}{Hybrid} & \multicolumn{2}{c|}{Business-oriented} \\ \hline
%     \multicolumn{1}{|c|}{\multirow{3}{*}{Technical}} & \multirow{2}{*}{Tasks} & \textit{Data tasks} & \multicolumn{1}{c|}{\begin{tabular}[c]{@{}c@{}}Descriptive \\  analysis\end{tabular}} & \multicolumn{1}{c|}{\begin{tabular}[c]{@{}c@{}}Predictive \\  analysis\end{tabular}} & \multicolumn{1}{c|}{\begin{tabular}[c]{@{}c@{}}Prescriptive \\  analysis\end{tabular}} & \multicolumn{2}{c|}{Tools} & \multicolumn{1}{c|}{\begin{tabular}[c]{@{}c@{}}Infra- \\  structure\end{tabular}} & \multicolumn{1}{c|}{Visualization} & Dashboards \\ \cline{3-11} 
%     \multicolumn{1}{|c|}{} &  & \textit{\begin{tabular}[c]{@{}c@{}}Business \\  tasks\end{tabular}} & \multicolumn{1}{c|}{\begin{tabular}[c]{@{}c@{}}Process \\  improvement\end{tabular}} & \multicolumn{2}{c|}{\begin{tabular}[c]{@{}c@{}}Decision-\\ making\end{tabular}} & \multicolumn{2}{c|}{New insights} & \multicolumn{2}{c|}{\begin{tabular}[c]{@{}c@{}}Products / \\  Services\end{tabular}} & Standardization \\ \cline{2-11} 
%     \multicolumn{1}{|c|}{} & \begin{tabular}[c]{@{}c@{}}Tech- \\  nology\end{tabular} & \textit{\begin{tabular}[c]{@{}c@{}}Data \\  repository\end{tabular}} & \multicolumn{2}{c|}{United} & \multicolumn{4}{c|}{Hybrid} & \multicolumn{2}{c|}{Dispersed} \\ \hline
%     \end{tabular}
%     \caption{undefined}
%     \label{undefined}
% \end{table}

\section{Experiment Evolution Model (Microsoft)}
\section{Conceptual requirements}
\section{DI / DS Integration Framework}
