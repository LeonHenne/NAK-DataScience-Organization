% insert table summary of CSPG
\begin{table}[htb]
    \centering
    \begin{tabular}{>{\raggedright\arraybackslash}m{2.5cm} >{\raggedright\arraybackslash}m{5cm} >{\raggedright\arraybackslash}m{5cm}}
    \toprule
     & \textbf{Department/Unit Level} & \textbf{Organizational Level} \\ \midrule
    \textbf{Analytics Culture} & Sind Big Data und Analytics als organisatorische Funktion betrachtet und gibt es eine Big Data/Analytics-Abteilung oder -Einheit, um diese Funktion zu unterstützen? & Sind Big Data und Analytics in die Unternehmensstrategie integriert? Gibt es einen Senior Leader, der sich für Big Data und Analytics einsetzt? Werden interne und externe Daten, die Wert schaffen können, genutzt? \\ \midrule
    \textbf{Analytics Staff} & Hat die Analytics-Abteilung die richtigen Personen mit dem richtigen Grad an analytischer Spezialisierung, IT-Wissen und Geschäftswissen? & Gibt es Analytikteammitglieder in den richtigen Abteilungen innerhalb der Organisation und gibt es eine kritische Masse an analytischem Talent? Wenn nicht, muss das analytische Personal neu ausbalanciert oder die Zentralisierung/Dezentralisierung des Analysepersonals entsprechend angepasst werden. \\ \midrule
    \textbf{Analytics Processes} & Hat die Analyseabteilung Analyseprozesse eingerichtet, um Analysemodelle zu erstellen, Analysemodelle einzusetzen und deren Geschäftsauswirkungen zu messen? & Hat die Organisation die analytischen Prozesse eingerichtet, um analytische Möglichkeiten auszuwählen, Daten für die Datenwissenschaftler bereitzustellen, analytische Modelle zu erstellen, analytische Modelle einzusetzen und den generierten Geschäftswert zu messen? Gibt es eine analytische Governance-Struktur, um die richtigen analytischen Prozesse zu unterstützen und zu koordinieren? \\ \bottomrule
    \end{tabular}
    \caption{Vergleich von Analytics auf Abteilungs-/Einheitsebene und organisatorischer Ebene}
\end{table}