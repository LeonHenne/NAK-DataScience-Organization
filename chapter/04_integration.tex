\chapter[Integrationsprozesse]{Integrationsprozesse}

Anschließend an die Beschreibung einer datengesteuerten Organisation, dessen Aufbau, Prozesse und Rollen werden innerhalb dieses Kapitels die Herausforderungen und Vorgehenswege zur Transformation in eine datengesteuerte Organisation thematisiert.

\section{Herausforderungen}

Bereits aus dem Zielbild, datengesteuert Entscheidungen in Organisation zu treffen (z.B. zur Strategie), ergibt sich die Herausforderung, Entscheidungen hohen Ausmaßes in sich schnell wechselnden kompetitiven Umgebungen auszuführen. \footcite[Vgl.][S. 2]{Pratt.2023} 
Hinzu zu der Geschwindigkeit der Geschäftsumgebung, konnte durch eine Gartner Umfrage festgestellt werden, dass 65 \% der Teilnehmenden im Zeitraum der letzten zwei Jahre (2021-2023) einen Zuwachs in der Komplexität der Entscheidungen verzeichneten. \footcite[Vgl.][S. 65]{Pratt.2023}

Daraus resultierend zeigt sich, dass vor allem managementbezogene und kulturelle Herausforderungen die Transformation beeinflussen, neben den ohnehin bestehenden technischen Hürden. \footcite[Vgl.][S. 15]{Dalpiaz.2020}
% Diese Herausforderungen bilden dabei jedoch recht umfassend die notwendigen Aspekte zur 

\begin{itemize}
    \item outside view (fast and complex decisions)
    \item manageral and cultural challenges
    \item technical challenges
    \item need for DS framework
\end{itemize}

\section{Vorgehensmodelle zur Integration von Data Science}

\begin{itemize}
    \item Conceptual requirements
    \item Design Parameters
    \item Experiment Evolution Model (Microsoft)
    \item CSPG Framework
    \item DI / DS Integration Framework
\end{itemize}