\chapter[Problemstellung]{Problemstellung}

Viele Trends der technologischen Welt der letzten Jahrzehnte wurde besonders durch drei bestimmende Faktoren beeinflusst.
Die Entwicklung der Datenspeicherungs- und die Entwicklung der Rechenkapazitäten als zwei dieser Faktoren entstammen den Erkenntnissen und sich bestätigenden Entwicklungen von Gorden E. Moore bzw. dem Moorschen Gesetz. \parencite[Vgl.][S. 1]{Moore.1998} % \footcite[Vgl.][S. 1]{Moore.1998}
Demnach wurde bereits sehr früh erkannt, dass integrierte Schaltungen die bekannten Telefonschaltungen ersetzen wird und Computer leistungsfähiger entsprechende Daten verarbeiten können. \footcite[Vgl.][S. 1]{Moore.1998}
Erkenntlich wird diese Entwicklung anhand der heutzutage konstanten und umfangreichen Generierung von Daten durch Anwendungen in Mobiltelefonen, Autos, IOT Geräten oder Industriemaschinen. \footcite[Vgl.][S 3f.]{Dalpiaz.2020}
Der dritte Faktor ist die Entwicklung der Möglichkeiten der Datenverarbeitung und Analyse anhand von zunehmend komplexeren Algorithmen. \footcite[Vgl.][S. 4]{Dalpiaz.2020}

Folgend aus diesen drei Faktoren zielen immer mehr Software Unternehmen darauf ab, sich in eine datengestützte Organisation zu transformieren. \footcite[Vgl.][S. 1]{Fabijan.2017}
Höheres Interesse dem Besitz und der Verarbeitung großer Datenmengen zeigen heutzutage jedoch Organisationen aller Bereiche wie Wirtschaft, Regierungen und Forschung. \footcite[Vgl.][S. 1]{Pratt.2023}
Deren Stratgien weisen häufig auf, mittels datenfokussierter Kultur und Datenanalysen möglichst umfangreiche faktenbasierte Entscheidungen zu treffen. \footcite[Vgl.][S. 18]{Dalpiaz.2020}
Forschungsfeld dieser Absicht bildet die Data Science, bzw. Big Data Analytics mit dessen Einsatz von Technologien sich Unternehmen einen Wettbewerbsvorteil erzielen. \footcite[Vgl.][S. 3]{Dalpiaz.2020}
Technologie getriebene Unternehmen können dadurch bis zu 26 \% profitabler werden lassen, als Wettbewerber, welche weniger bis gar keine digitalen Technologien einsetzen. \footcite[Vgl.][S. 1]{Fabijan.2017}
Ein Grund dafür können bessere und schnellere Entscheidungen sein, welche aus dem vollumfänglicheren Verständnis des Kunden und höherer Transparenz des Entwicklungsprozesses resultieren. \footcite[Vgl.][S. 18]{Dalpiaz.2020}

Trotz der heute vorhandenen Fülle von erzeugten Daten bleibt die Anzahl an Unternehmen, welche erfolgreich in eine datengestützte Organisation transformierten eher gering. \footcite[Vgl.][S. 1]{Fabijan.2017}
Zusätzlich existiert bisher nur wenig Forschung dazu, wie solche Transformationen zu datengetriebenen Unternehmen umzusetzen sind. \footcite[Vgl.][S. 1]{Fabijan.2017}

% \begin{itemize}
%     \item Entwicklung von Datenspeicherung, Rechenpower und Analysemöglichkeiten
%     \item Relevanz der Thematik für Organisationen als Wettbewerbsvorteil (was bringt es ?) 
%     \begin{itemize}
%         \item Steigende Notwendigkeit der Integration von Data Science zur Marktpositionierung
%     \end{itemize}
%     \item Aktualität des Problems (Bisher hat es halt keiner gemacht)
% \end{itemize}