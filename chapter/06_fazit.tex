\chapter[Fazit]{Fazit}
Ziel dieser Arbeit war die Beleuchtung der aktuellen Forschungsliteratur zur Integration von Data Science in Organisationen und Abteilungen.
Dazu wurde zunächst ein Grundlagenverständnis der Data Science geschaffen und das Zielbild einer datengesteuerten Organisation thematisiert.
Anschließend wurde im Detail auf die Herausforderungen der Integration eingegangen.
Folgend wurden aus der Forschungsliteratur verschiedene Modelle zur Transformation in eine datengesteuerte Organisation und dessen Gestaltung dargelegt.

Durch die Hausarbeit konnte verdeutlicht werden, welche Auswirkung eine Integration der Data Science im Unternehmen erzeugen kann, jedoch auch wie herausfordernd und aufwändig eine Transformation zur datengesteuerten Organisation ist.
Als Disziplin um aus Daten Mehrwerte zu generieren kann die Data Science in datengesteuerten Organisationen Prozesse verbessern, transparentere Entscheidungsgrundlagen schaffen und Innovationen fördern.
Externe Faktoren, wie das sich schnell verändernde Umfeld und interne Herausforderungen, wie die Datenkultur, das Vertrauen in Daten und fehlende IT-Infrastruktur erschweren jedoch die Integration der Data Science.
In der Forschung wurden daher verschiedene Modelle entwickelt um die Integration phasenweise zu begleiten, oder das Zielbild in leichter umzusetzende Teilaspekte aufzugliedern. 

Zusammenfassend lässt sich erkennen, dass die Integration der Data Science in Organisation eine wichtige Aufgabe für alle Teile der Gesellschaft wie Wirtschaft, Politik und Forschung geworden ist.
Daher sollten alle Gesellschaftsbereiche mit Best Practices, Bildungsmaßnahmen und ausgiebiger Forschung zur Bewältigung dieser Aufgabe beitragen.