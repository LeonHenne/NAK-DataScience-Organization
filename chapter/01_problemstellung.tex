\chapter[Problemstellung]{Problemstellung}

Viele Trends der technologischen Welt der letzten Jahrzehnte wurde besonders durch drei bestimmende Faktoren beeinflusst.
Die Entwicklung der Datenspeicherungs- und die Entwicklung der Rechenkapazitäten als zwei dieser Faktoren entstammen den Erkenntnissen und sich bestätigenden Entwicklungen von Gorden E. Moore bzw. dem Moorschen Gesetz. \footcite[prenote][postnote]{(By 1975)}
Demnach wurde bereits sehr früh erkannt, dass integrierte Schaltungen die bekannten Telefonschaltungen ersetzen wird und Computer leistungsfähiger entsprechende Daten verarbeiten können. \footcite[prenote][postnote]{(Integrated circuits)}
Erkenntlich wird diese Entwicklung anhand der heutzutage konstanten und umfangreichen Generierung von Daten durch Anwendungen in Mobiltelefonen, Autos, IOT Geräten oder Industriemaschinen. \footcite[prenote][postnote]{constant generation}
Der dritte Faktor ist die Entwicklung der Möglichkeiten der Datenverarbeitung und Analyse anhand von zunehmend komplexeren Algorithmen. \footcite[prenote][postnote]{(significant evolution)}

Folgend aus diesen drei Faktoren zielen immer mehr Software Unternehmen darauf ab, sich in eine datengestützte Organisation zu transformieren. \footcite[prenote][postnote]{(Software development companies)}
Höheres Interesse dem Besitz und der Verarbeitung großer Datenmengen zeigen heutzutage jedoch Organisationen aller Bereiche wie Wirtschaft, Regierungen und Forschung.
Deren Stratgien weisen häufig auf, mittels datenfokussierter Kultur und Datenanalysen möglichst umfangreiche faktenbasierte Entscheidungen zu treffen. \footcite[prenote][postnote]{(general trend)}
Forschungsfeld dieser Absicht bildet die Data Science, bzw. Big Data Analytics mit dessen Einsatz von Technologien sich Unternehmen einen Wettbewerbsvorteil erzielen. \footcite[prenote][postnote]{(organisations are turning)}
Technologie getriebene Unternehmen können dadurch bis zu 26 \% profitabler werden lassen, als Wettbewerber, welche weniger bis gar keine digitalen Technologien einsetzen. \footcite[prenote][postnote]{(DDOs are more profitable)}

Trotz der heute vorhandenen Fülle von erzeugten Daten bleibt die Anzahl an Unternehmen, welche erfolgreich in eine datengestützte Organisation transformierten eher gering.
Zusätzlich existiert bisher nur wenig Forschung dazu, wie solche Transformationen zu datengetriebenen Unternehmen umzusetzen sind. \footcite[prenote][postnote]{(despite having data)}

% \begin{itemize}
%     \item Entwicklung von Datenspeicherung, Rechenpower und Analysemöglichkeiten
%     \item Relevanz der Thematik für Organisationen als Wettbewerbsvorteil (was bringt es ?) 
%     \begin{itemize}
%         \item Steigende Notwendigkeit der Integration von Data Science zur Marktpositionierung
%     \end{itemize}
%     \item Aktualität des Problems (Bisher hat es halt keiner gemacht)
% \end{itemize}